\documentclass{beamer}

% Theme selection
\usetheme{metropolis}           % Use metropolis theme
\metroset{background=light} % Optional: 'dark' or 'light' background. Let's stick to default (light) for safer printing, or dark if "attractive" means sleek. Let's try standard first, maybe add some color configuration.
\metroset{block=fill}

\usepackage{xcolor}
\definecolor{spitred}{RGB}{179, 0, 0} % Deep red for SPIT
\setbeamercolor{frametitle}{bg=spitred}
\setbeamercolor{alerted text}{fg=spitred}

\setbeamertemplate{navigation symbols}{} % Remove navigation symbols
\setbeamertemplate{caption}[numbered]

\usepackage{graphicx}
\usepackage{booktabs}
\usepackage{amsmath}
\usepackage{algorithm}
\usepackage{algorithmic}

% --- Background Configuration ---
% Define a command to switch to the "Content" background (no-header.png)
% Define a command to switch to the "Content" background (no-header.png)
\newcommand{\setcontentbg}{
    \usebackgroundtemplate{
        \includegraphics[width=\paperwidth,height=\paperheight]{no-header.png}
    }
}

% Define a command to switch to the "Plain" background (bgnew.jpeg)
\newcommand{\setplainbg}{
    \usebackgroundtemplate{
        \includegraphics[width=\paperwidth,height=\paperheight]{bgnew.jpeg}
    }
}

% Set default to Content background
\setcontentbg

% Hook to inject Section Slides with Plain background
\AtBeginSection[]{
  {
    \setplainbg
    \begin{frame}[standout]
      \sectionpage
    \end{frame}
  }
}


% Metadata
\title[Enhancing QoS in IEEE 802.11ax]{Enhancing QoS in Dense IEEE 802.11ax Networks using a Dynamic Airtime-Based Soft Admission Control Mechanism}
\author[Ambawade, Pawar]{Dayanand Ambawade\inst{1} \and Rohan Pawar\inst{2}}
\institute[SPIT]{
  \footnotesize
  \inst{1} Department of Computer Science and Engineering\\
  \inst{2} Department of Electronics \& Telecommunication Engineering\\[-0.5em]
  Sardar Patel Institute of Technology (SPIT)\\
  Mumbai, India
}
\date[ICSCC 2025]{10th International Conference on Systems, Control and Communications (ICSCC 2025)\\Nagoya University, Japan}

\begin{document}

% Slide 1: Title
{
\setplainbg
\begin{frame}
  \vspace{0.3cm}
  \titlepage
\end{frame}
}

% Slide 2: Outline
{
\setplainbg
\begin{frame}{Outline}
  \tableofcontents
\end{frame}
}

% ... (some lines skipped in context, handled by AllowMultiple logic if needed, but I'll focus on the specific frames)

% Correction for Comprehensive Analysis Frame
\begin{frame}{Comprehensive Analysis}
  \begin{figure}
    \centering
    \includegraphics[width=0.85\linewidth,height=0.75\textheight,keepaspectratio]{../results/graphs/ascac_comprehensive_4panel.png}
    \caption{Throughput, Latency, Utilization, and Flow Count}
  \end{figure}
\end{frame}


% Slide 3: Introduction
\section{Introduction}
\begin{frame}{Motivation}
  \begin{itemize}
    \setlength\itemsep{1em}
    \item \textbf{The Challenge:} IEEE 802.11ax (Wi-Fi 6) struggles under saturation in dense environments.
    \item \textbf{High Latency:} Real-time apps (VoIP) suffer >45ms delays without control.
    \item \textbf{Existing Solutions Fail:}
    \begin{itemize}
        \item Count-based CAC ignores heterogeneity.
        \item Static thresholds waste capacity.
    \end{itemize}
    \item \textbf{Goal:} Maximize airtime utilization while guaranteeing strict QoS.
  \end{itemize}
\end{frame}

% New Slide: Test Bed / Components
\section{System Architecture}
\begin{frame}{Research Test Bed \& Components}
  \begin{columns}
    \column{0.5\textwidth}
      \textbf{Simulation Components (ns-3):}
      \begin{itemize}
          \item \textbf{AP Node:} Wi-Fi 6 (802.11ax), 80 MHz, 5 GHz.
          \item \textbf{Stations:} 25-50 users in dense grid.
          \item \textbf{Traffic Generators:}
          \begin{itemize}
              \item VoIP (UDP)
              \item Video (UDP)
              \item Bursty/Web (TCP/UDP)
          \end{itemize}
      \end{itemize}
      
    \column{0.5\textwidth}
      \begin{figure}
        \centering
        \includegraphics[width=0.9\linewidth]{../results/graphs/ns3_network_topology.png}
        \caption{NS-3 Network Topology Visualization}
      \end{figure}
  \end{columns}
\end{frame}

\begin{frame}{System Model and Traffic}
   \begin{columns}
     \column{0.5\textwidth}
       \textbf{Metric: Airtime Utilization}
       \begin{equation*}
          \alpha_c = \frac{R_c}{\eta \cdot R_{phy}^c}
       \end{equation*}
       
       \vspace{1em}
       \textbf{Traffic Mix:}
       \begin{itemize}
           \item \textbf{VoIP:} High Priority, Low Bandwidth.
           \item \textbf{Video:} Med Priority, High Bandwidth.
           \item \textbf{Best Effort:} Low Priority, Bursty.
       \end{itemize}
       
     \column{0.5\textwidth}
       \begin{figure}
         \centering
         \includegraphics[width=0.9\linewidth]{../results/graphs/ap_station_distribution.png}
         \caption{AP and Station Distribution}
       \end{figure}
   \end{columns}
\end{frame}

% Section: Proposed Solution / Algorithms
\section{Proposed AS-CAC Framework}

\begin{frame}{Proposed Solution: Soft CAC}
  \textbf{Concept:} Priority-Aware Thresholds.
  
  \begin{table}
    \centering
    \begin{tabular}{lcc}
      \toprule
      \textbf{Traffic Class} & \textbf{Priority} & \textbf{Threshold ($\theta_c$)} \\
      \midrule
      VoIP (AC\_VO) & High & 90\% \\
      Video (AC\_VI) & Medium & 80\% \\
      Best Effort (AC\_BE) & Low & \textbf{95\%} \\
      \bottomrule
    \end{tabular}
  \end{table}
  
  \begin{figure}
      \centering
      \includegraphics[width=0.7\linewidth]{../results/graphs/soft_vs_hard_cac_comparison.png}
      \caption{Soft vs Hard CAC Thresholding}
  \end{figure}
\end{frame}

\begin{frame}{Algorithm: AS-CAC+ (Adaptive)}
  \textbf{Dynamic Threshold Adjustment:}
  \begin{itemize}
      \item Monitors Packet Error Rate (PER) and Utilization.
      \item Adjusts Best-Effort threshold ($\theta_{BE}$) in real-time.
  \end{itemize}
  
  \vspace{0.5em}
  
  \begin{algorithm}[H]
    \footnotesize
    \caption{AS-CAC+ Adaptive Threshold}
    \begin{algorithmic}[1]
      \STATE \textbf{Input:} Utilization $A$, Threshold $\theta_{BE}$
      \STATE $PER \gets$ Calculate from Network Health
      \IF{$PER > 0.05$}
          \STATE $\theta_{BE} \gets \max(0.80, \theta_{BE} - 0.01)$ \COMMENT{reduce load}
      \ELSIF{$PER < 0.02$ AND $A > 0.70$}
          \STATE $\theta_{BE} \gets \min(0.98, \theta_{BE} + 0.01)$ \COMMENT{utilize spare capacity}
      \ENDIF
      \STATE \textbf{Return:} Updated $\theta_{BE}$
    \end{algorithmic}
  \end{algorithm}
\end{frame}

\begin{frame}{Adaptive Behavior Visualization}
    \begin{figure}
        \centering
        \includegraphics[width=0.85\linewidth]{../results/graphs/ascac_threshold_evolution.png}
        \caption{Dynamic Adaptation of Thresholds over Time}
    \end{figure}
\end{frame}


% Section: Results
\section{Performance Evaluation}

\begin{frame}{Simulation Results: Latency}
  \begin{figure}
    \centering
    \includegraphics[width=0.8\linewidth]{../results/graphs/cac_vs_no_cac.png}
    \caption{Impact on Latency: No CAC vs AS-CAC}
  \end{figure}
\end{frame}



\begin{frame}{Multi-Dimensional Superiority}
  \begin{columns}
    \column{0.5\textwidth}
      \begin{figure}
        \centering
        \includegraphics[width=\linewidth]{../results/graphs/ascac_radar_chart.png}
      \end{figure}
    \column{0.5\textwidth}
      \textbf{Why AS-CAC+ Wins:}
      \begin{itemize}
          \item \textbf{Adaptability:} Reacts to interference.
          \item \textbf{Utilization:} 97.4\% vs 78\% (Hard).
          \item \textbf{Safety:} Keeps VoIP < 2ms.
      \end{itemize}
  \end{columns}
\end{frame}

% Section: Conclusion
\section{Conclusion}
\begin{frame}{Conclusion}
  \begin{block}{Summary}
    \begin{itemize}
        \item \textbf{AS-CAC+} transforms admission control from static to dynamic.
        \item It safely unlocks \textbf{19.2\% more capacity}.
    \end{itemize}
  \end{block}
  
  \vspace{1em}
  
  \begin{center}
      \textbf{Thank You!}
      \\
      \vspace{1em}
      \small{Dayanand Ambawade, Rohan Pawar}
  \end{center}
\end{frame}

\end{document}
